\documentclass[a4paper]{article}

\usepackage[T1]{fontenc}
\usepackage[utf8]{inputenc}
\usepackage[english]{babel}
\usepackage{csquotes}
\usepackage{listings}
\usepackage{multicol}
\lstset{language=c,frame=single,captionpos=b}
\usepackage{hyperref}
\usepackage{amsmath}
\usepackage[backend=biber, sorting=none,maxbibnames=40]{biblatex}
\renewbibmacro{in:}{}
\addbibresource{ref.bib}
\usepackage{graphicx}
\usepackage{placeins}
\usepackage[margin=2.5cm]{geometry}
\usepackage{subcaption}
\usepackage[affil-it]{authblk}
\usepackage{color}
\usepackage{amssymb,amsmath}
\usepackage{subfloat}
\usepackage{float}

\begin{document}
\title{GPU Assignment: Image processing}
\author{Stefano Sandonà}
\affil{Vrije Universiteit Amsterdam, Holland}
\date{}
		
\maketitle

\section{GPUs: NVIDIA GTX480}
\label{sec:nvidia}
The aim of this assignment was to learn how to use many-core accelerators, GPUs in this particular case, to parallelize data-intensive code. All the implementations were written for the \textbf{NVIDIA GTX480}, using CUDA, a parallel computing platform and programming model invented by NVIDIA. Programming with CUDA, there is a straightforward mapping onto hardware, for this reason it is necessary to study the available HW before start developing an application. The architecture of the given accelerator is is shown in Figure \ref{fig:gtx}, its main characteristics and limits are shown in Table \ref{table:t1}.

\begin{figure}[ht]
    \centering
    \includegraphics[width=0.7\linewidth]{gtx}
    \caption{NVIDIA GTX480 Architecture}
    \label{fig:gtx}
\end{figure}
\FloatBarrier

\begin{table}[ht]
\centering
\begin{tabular}{l|l}
Microarchitecture & Fermi \\ \hline
Compute capability (version) & 2.0 \\ \hline
Maximum dimensionality of grid of thread blocks & 3 \\ \hline
Maximum x-dimension of a grid of thread blocks & 65535 \\ \hline
Maximum y-, or z-dimension of a grid of thread blocks & 65535 \\ \hline
Maximum dimensionality of thread block & 3 \\ \hline
Maximum x- or y-dimension of a block & 1024 \\ \hline
Maximum number of threads per block & 1024 \\ \hline
Cores per SM (warp size) & 32 \\ \hline
SM & 15 \\ \hline
Cores & 480 (32 * 15) \\ \hline
Maximum number of resident blocks per multiprocessor & 8 \\ \hline
Maximum number of resident warps per multiprocessor & 48 \\ \hline
Maximum number of resident threads per multiprocessor & 1536 (48 * 32) \\ \hline
Number of 32-bit registers per multiprocessor & 32K \\ \hline
Maximum amount of shared memory per multiprocessor & 48K
\end{tabular}
\caption{NVIDIA GTX480 Specifications}
\label{table:t1}
\end{table}


\section{CImg}
\label{sec:cimg}
The image processing library used in this project was CImg, a small, modern and open-source toolkit developed for C++. CImg implements the RGB color model, an additive color model in which red, green, and blue light are added together in various ways to reproduce a broad array of colors. Each colored image of size \textit{N*M} is composed by three parts (R,G,B) of the same size, so that \textit{N*M*3} values are necessary to define an image. The Figure \ref{fig:rgb} shows an example of image composition.

\begin{figure}[ht]
    \centering
    \includegraphics[width=0.7\linewidth]{rgb}
    \caption{RGB model}
    \label{fig:rgb}
\end{figure}
\FloatBarrier

\section{The processing flow}
\label{sec:cpf}
Using CUDA there are two parts of the code: the device code, or GPU code, or the Kernel, that is a sequential program, write for one thread and execute for all and the HOST code, or CPU code, that is used to instantiate the grid, run the kernel, manage the memory. Figure \ref{fig:flow} shows the processing flow of a CUDA application. In the particular case of image processig, everything starts from the CPU, that store the image from a file into a local buffer, allocates IN and OUT buffers on the GPU (\textit{cudaMalloc}) and copy the image into the GPU's IN buffer (\textit{cudaMemCpy}). After that, the CPU launches the GPU kernel with a defined grid configuration (\textit{kernel\_function <<gridDim, blockDim>>(params)}), that is executed by the GPU following the SIMT (Single Instruction, Multiple Threads) NVIDIA model. The threads are executed in parallel in each core, and they read the assigned part of IN data and generates the assigned part of OUT data. At the end, the results are copied out back to the CPU (\textit{cudaMemCpy}) and the image is written to a file by the CPU.

\begin{figure}[ht]
    \centering
    \includegraphics[width=0.5\linewidth]{flow}
    \caption{CUDA processing flow}
    \label{fig:flow}
\end{figure}
\FloatBarrier

\section{CUDA grid configuration}
\label{sec:grid}
In CUDA, as mentioned before, there is a strinct mapping with the hardware, so that an hardware virtualization model is fixed with the concepts of thread, block and grid.
Each \textbf{thread} executes the kernel code, running on one CUDA core.
The threads are logically grouped into \textbf{thread blocks}, so that the threads of the same block will run on the same multiprocessor. The thread blocks are logically organized in a \textbf{Grid}, that represent the entire dataset. The blocks and the grid can be of 1D, 2D or 3D.
The most important thing programming with GPUs, to make use of all their power, is to make them as busy as possible. Switching between concurrent warps has no overhead because registers and share memory are partitioned and not stored/restored, so that if one warp has to wait for example for a memory access and another warp is ready, there is a switch to hide the latency. The thread scheduling is really quick, and this allow the blocks to be swapped in and out really quickly. For this reason an high number of warps is needed and instantiating a grid, is good to have a number of blocks much bigger than the number of available multiprocessor and a block size that can be higher than the amount of cuda cores available per SM.
Another interesting aspect to take into consideration, is the block size. The block is divided into warps, so that Threads 0..31 are part of Warp 0, Threads 32...63 are part of warp 1 and so on. Considering this, it is good to have blocks with a size multiple of 32, so that no useless threads are launched. After certain tests, a block of size 256 revealed to be optimal. For what concerning the grid configuration, an image is a 2D structure and for this reason it is intuitive to set up also a 2D grid. By the way, setting up a 2D grid, there is not only one way to follow. For this particular project two possible block configurations were considered, 1D (1x256) and 2D (16x16) and two possible grid configurations: square and non square. In the rest part of this document, \textbf{M1} indicates a non square grid of 1D blocks (Figure \ref{fig:ns1}), in which the grid has a height of \textit{image\_height} and a width of \textit{(ceil(image\_width / 256))}; \textbf{M2} indicates a square grid of 1D blocks (Figure \ref{fig:s1}), in which the size of the grid is \textit{ceil(sqrt((image\_width * image\_height)/256))}; \textbf{M3} indicates a non square grid of 2D blocks (Figure \ref{fig:ns2}), in which the grid has a width of \textit{ceil(image\_width / 16)} and a height of \textit{ceil(image\_height / 16)}; \textbf{M4} indicates a square grid of 2D blocks (Figure \ref{fig:s2}), in which the size of the grid is the same as M2. 


\begin{figure}[!ht]
\begin{subfigure}{0.5\textwidth}
\centering
\includegraphics[width=\linewidth]{res/1D_no_square}
\caption{No square grid with 1D blocks}
\label{fig:ns1}
\end{subfigure} % separation between the subfigures
\begin{subfigure}{0.5\textwidth}
\centering
\includegraphics[width=\linewidth]{res/1D_square}
\caption{Square grid with 1D blocks}
\label{fig:s1}
\end{subfigure}
 % separation between the subfigures
\begin{subfigure}{0.5\textwidth}
\centering
\includegraphics[width=\linewidth]{res/2D_no_square}
\caption{No square grid with 2D blocks}
\label{fig:ns2}
\end{subfigure}
 % separation between the subfigures
\begin{subfigure}{0.5\textwidth}
\centering
\includegraphics[width=\linewidth]{res/2D_square}
\caption{Square grid with 2D blocks}
\label{fig:s2}
\end{subfigure}
\caption{Possible kernel configurations}
 \label{fig:speed}
\end{figure}
\FloatBarrier

\section{Coalesced memory access}
\label{sec:cma}
One of the main bottlenecks with the GPUs are the global memory accesses that are expensive, for this reason it is better to maximize the use of bytes that travel from the DRAM to the Streaming Multiprocessor. CUDA uses a SIMT approach, in which all threads of a warp execute the same instruction, so that global memory accesses are effectuated "per warp". The threads in a warp (32) provide 32 addresses and the hardware converts these addresses into memory transactions. The memory is divided into regions of 128 bytes, so that bytes 0...127 are part of Region 0, bytes 128...255 are part of Region 1 and so on. The memory is accessed per region, that means if one thread wants the byte 0, the entire Region 0 is loaded. A kernel is correctly designed, if consecutive threads access consecutive memory addresses, so that all the requeted addresses fall on the same region and only one transaction is performed. Behaving in this "coalesced way" instead of having one access per thread, these accesses are grouped and the total memory overhead is reduced. Developing all the three algorithms this aspect was taken into consideration.

\section{More work per thread}
\label{sec:mwpt}
The standard (and easyest) way to proceed on image computing is to use a grid large enough to cover all the image and than compute one thread per pixel. An aspect that is useful to consider establishing the amount of work per thread is the Arithmetic Intensity (AI) of the board. In fact, if we choose a number of operations per thread that allows to have a decent AI, that is, an AI that matches the one of the card, an increase of the performance can be obtained. Rea

Check if darker uni is memory bound

\# FLOP / byte -> arithmetic intensity (memory or compute bound)

\section{Algorithm 1: Grayscale Conversion and Darkening}
\label{sec:gcd}
From an RGB image, the output of this algorithm is a darker grayscale image.
The gray value of a pixel is generated by weighting the three values (\texttt{0.3*R, 0.59*G, 0.11*B}) and then summing them together. To darken the obtained grayscale image, the final pixel value is multiplied by a constant (0.6). The Figure \ref{fig:dark} shows an example of the result. The sequential algorithm, simply go through the entire image and computes for each pixel the corresponding value (Listing \ref{dsc}).

\begin{figure}[ht]
    \centering
    \includegraphics[width=\linewidth]{dark}
    \caption{Grayscale Conversion and Darkening}
    \label{fig:dark}
\end{figure}
\FloatBarrier

\begin{lstlisting}[label=dsc, caption=Darker Sequential code]
//H=image_height, W=image_width
for ( int y = 0; y < H; y++ ) {
 for ( int x = 0; x < W; x++ ) {
  float grayPix = 0.0f;
  float r = static_cast< float >(inputImage[(y * W) + x]);
  float g = static_cast< float >(inputImage[(W * H) + (y * W) + x]);
  float b = static_cast< float >(inputImage[(2 * W * H) + (y * W) + x]);
  grayPix = ((0.3f * r) + (0.59f * g) + (0.11f * b));
  grayPix = (grayPix * 0.6f) + 0.5f;
  darkGrayImage[(y * W) + x] = static_cast< unsigned char >(grayPix);
 }
}
\end{lstlisting}
\FloatBarrier

\subsection{Parallelization}
\label{sec:p1}
\subsection{First method}
\label{sec:dfm}
After allocating into the GPU global memory some memory to contain the input image (\textit{3 * image\_width * image\_height * sizeof(unsigned char)}), copying the input image there and allocating some memory to content the output image (\textit{image\_width * image\_height * sizeof(unsigned char)}), the kernel is ready to be launched. The GPU code is the same as the code content inside the loop of the sequential version (Listing \ref{dsc}), but instead of using the indexes of the loop to access the image pixels, it uses the index associated to the thread. To exploit the coalesced memory access, two consecutive threads computes/accesses the values of two consecutive pixels. The 4 different grid configurations (explained in Section \ref{sec:grid}) were tested for this program, in order to establish the best. In Figure \ref{fig:histo_darker} are reported the obtained results. As shown methods M1 and M2 obtained the best results. 


\begin{figure}[ht]
    \centering
    \includegraphics[width=0.9\linewidth]{res/darker_histo}
    \caption{Speedups Comparison}
    \label{fig:histo_darker}
\end{figure}
\FloatBarrier


Uni blocks, overflow (more) - M1 more
\begin{figure}[ht]
    \centering
    \includegraphics[width=0.9\linewidth]{res/darker_try_histo}
    \caption{Normalized Speedup}
    \label{fig:norm_histo_darker}
\end{figure}
\FloatBarrier

\subsection{Optimization - More pixels per thread}
\label{sec:dfm}

Uni blocks, square grid (more) - M2 more
\begin{figure}[ht]
    \centering
    \includegraphics[width=0.9\linewidth]{res/darker_square_more_histo}
    \caption{Normalized Speedup}
    \label{fig:norm_histo_darker}
\end{figure}
\FloatBarrier

\begin{figure}[ht]
    \centering
    \includegraphics[width=0.9\linewidth]{res/darker_best_comparison}
    \caption{Normalized Speedup}
    \label{fig:norm_histo_darker}
\end{figure}
\FloatBarrier

\begin{lstlisting}[label=dpc, caption=Darker Parallel Code]
for(i = ((blockIdx.y * gridDim.x + blockIdx.x) * blockDim.x) + threadIdx.x; 
    i < width * height; 
    i += (gridDim.x * blockDim.x) * (gridDim.y * blockDim.y)) {
float grayPix = 0.0f;
float r = static_cast< float >(inputImage[i]);
float g = static_cast< float >(inputImage[(width * height) + i]);
float b = static_cast< float >(inputImage[(2 * width * height) + i]);
grayPix = ((0.3f * r) + (0.59f * g) + (0.11f * b));
grayPix = (grayPix * 0.6f) + 0.5f;
outputDarkGrayImage[i] = static_cast< unsigned char >(grayPix);
}
\end{lstlisting}
\FloatBarrier

Theoretical Throughput = 1345
Theorretical Bandwidth = 177.4

\begin{table}[]
\centering
\caption{My caption}
\label{my-label}
\begin{tabular}{|c|c|c|c|c|}
\hline
               & \textbf{\begin{tabular}[c]{@{}c@{}}Achieved Throughput\\ (GFLOPS/s)\end{tabular}} & \textbf{\begin{tabular}[c]{@{}c@{}}Achieved Bandwidth \\ (GB/s)\end{tabular}} & \textbf{\begin{tabular}[c]{@{}c@{}}Compute utilization \\ (\%)\end{tabular}} & \textbf{\begin{tabular}[c]{@{}c@{}}Bandwidth utilization \\ (\%)\end{tabular}} \\ \hline
\textbf{Img00} & 24,14                                                                             & 13,80                                                                         & 1,80                                                                         & 7,78                                                                           \\ \hline
\textbf{Img01} & 95,08                                                                             & 54,33                                                                         & 7,07                                                                         & 30,63                                                                          \\ \hline
\textbf{Img02} & 101,20                                                                            & 57,83                                                                         & 7,52                                                                         & 32,60                                                                          \\ \hline
\textbf{Img03} & 136,02                                                                            & 77,72                                                                         & 10,11                                                                        & 43,81                                                                          \\ \hline
\textbf{Img04} & 129,48                                                                            & 73,99                                                                         & 9,63                                                                         & 41,71                                                                          \\ \hline
\textbf{Img05} & 137,11                                                                            & 78,35                                                                         & 10,19                                                                        & 44,16                                                                          \\ \hline
\textbf{Img06} & 141,63                                                                            & 80,93                                                                         & 10,53                                                                        & 45,62                                                                          \\ \hline
\end{tabular}
\end{table}

\section{Algorithm 2: Histogram Computation}
\label{sec:hc}
From an RGB image, the output of this algorithm is a grayscale image with the relative histogram of 256 possible values of gray.
The histogram measures how often a value of gray is used in an image. The sequential algorithm, simply go through the entire image, computing for each pixel the corresponding gray value and incrementing the corresponding counter. The Figure \ref{fig:histo} shows an example of the result.

\begin{lstlisting}[label=loop2, caption=Sequential code]
for ( int y = 0; y < height; y++ ) {
		for ( int x = 0; x < width; x++ ) {
			float grayPix = 0.0f;
			float r = static_cast< float >(inputImage[(y * width) + x]);
			float g = static_cast< float >(inputImage[(width * height) + (y * width) + x]);
			float b = static_cast< float >(inputImage[(2 * width * height) + (y * width) + x]);

			grayPix = ((0.3f * r) + (0.59f * g) + (0.11f * b)) + 0.5f;

			grayImage[(y * width) + x] = static_cast< unsigned char >(grayPix);
			histogram[static_cast< unsigned int >(grayPix)] += 1;
		}
}
\end{lstlisting}

\begin{figure}[ht]
    \centering
    \includegraphics[width=0.7\linewidth]{histo}
    \caption{Histogram Computation}
    \label{fig:histo}
\end{figure}
\FloatBarrier

\subsection{Parallelization}
\label{sec:p2}
\subsection{First method}
\label{sec:fm2}
After copying the input image and the initial empty histogram into the GPU global memory and allocating some memory to content the output image, the kernel is ready to be launched. As for the previous algorithm, the GPU code is the same as the code content inside the loop of the sequential version (Figure \ref{loop2}), but instead of using the indexes of the loop to access the image pixels, it uses the index associated to the thread. Also in this case, to exploit the coalesced memory access, two consecutive threads computes/accesses the values of two consecutive pixels. The interesting aspect of the histograms, is that it is possible that two threads read at the same time read the same value of gray for a pixel, so that at the same time they try to increment the same corresponding histogram bin. For a correct execution, to avoid wrong updates, the increment has to be an atomic operation, so that only one thread at the time will modify the bin value. 

The logic behind an atomic operations is that each thread "locks" the variable, modify it, and "unlocks" it, so that the other threads that try to modify the same variable has to wait that this will be "unlocked". The worst case, in the histogram scenario, is a monocromo image, in which all the threads try to modify the same bin, which implies a long waiting queue of threads.
 Different grid configurations were tested launching this program. 
 
 
 What you have shown in your question is only correct for certain block sizes. Your "coalesced" access:

int i = blockIdx.x * blockDim.x + threadIdx.x;
float vx = in\_vector[i];
will result in coalesced access of in\_vector from global memory only when blockDim.x is greater than or equal to 32. Even in the coalesced case, each thread within a block which shares the same threadIdx.x value reads the same word from global memory, which seems to be counter-intuitive and wasteful.

Try histo with blocks 8x32
\section{Algorithm 3: Smoothing}
\label{sec:smoo}
Smoothing is the process of removing noise from an image by the means of statistical analysis. To remove the noise, each point is replaced by a weighted average of its neighbours. In this way small-scale structures are removed from the image. In this case a two-dimensional 5-point triangular smooth filter was used. The Figure \ref{fig:smooth} shows an example of the result.

\begin{figure}[ht]
    \centering
    \includegraphics[width=0.5\linewidth]{smooth}
    \caption{Smoothing}
    \label{fig:smooth}
\end{figure}
\FloatBarrier

\subsection{Parallelization}
\label{sec:p2}
This particular algorithm deals with square areas of the image (filter), so that using 2D blocks, the threads can efficiently share memory and prevent a lot of global memory accesses.

\begin{figure}[!ht]
\begin{subfigure}{0.5\textwidth}
\centering
\includegraphics[width=\linewidth]{filter}
\caption{No square grid with 1D blocks}
\label{fig:f1}
\end{subfigure} % separation between the subfigures
\begin{subfigure}{0.5\textwidth}
\centering
\includegraphics[width=\linewidth]{filter_letters}
\caption{Square grid with 1D blocks}
\label{fig:f2}
\end{subfigure}
\caption{Possible kernel configurations}
 \label{fig:speed}
\end{figure}
\FloatBarrier

\section{Algorithms analysis with the Visual Profiler}
\label{sec:vp}

Instruction-level parallelism (ILP) is a measure of how many of the operations in a computer program can be performed simultaneously. The potential overlap among instructions is called instruction level parallelism.

\subsection{Application 1}
\label{sec:a1}
After collecting the profile of the applications using \textbf{nvprof}, the output files were evaluated using the \textbf{Nvidia Visual Profiler}. The choosen configuration, uses unidimensional \textit{Blocks} of 256 threads, 12 \textit{Registers} and 0 \textit{Shared Memory}. For all the images the profiler found no issues for \textit{Divergent Execution} (threads that follow different if branches) and a \textit{Warp Execution Efficiency} (ratio of the average active threads per warp to the maximum number of threads per warp supported on a multiprocessor) of 100\%. This last value comes from the fact that is used a block of 256 threads, that is a multiple of 32 (warp size), so no useless threads are launched and from the fact that there is no thread divergent execution, so the threads within a warp can execute in a SIMD way, avoiding inactive threads within the warp. Examples of occupancy results are shown in Figure, all the the images, obtained an occupancy over 91\% except the first that obtained 82.5\%. This is probably correlated to its small size, but in any case the occupancy is not limiting the performance of the application. For all the images exept the 5th, the Profiler found no issues related to Global Memory Access Pattern. The 5th image, is the only one of the given set that has an amount of pixels that is not a multiple of 128, the memory in the device is allocated with a 128-byte line granularity, so probably, the addresses fall within 2 cache lines, so 2 transaction insteads of one and so a decrease of the memory bandwidth utilization (Figure )

\begin{figure}[ht]
    \centering
    \includegraphics[width=0.7\linewidth]{profiling/darker/darker_occupancy_00}
    \caption{Img00}
    \label{fig:occ00}
\end{figure}
\FloatBarrier

\begin{figure}[ht]
    \centering
    \includegraphics[width=0.7\linewidth]{profiling/darker/darker_06_occupancy}
    \caption{Img05}
    \label{fig:occ06}
\end{figure}
\FloatBarrier

\begin{figure}[ht]
    \centering
    \includegraphics[width=0.7\linewidth]{profiling/darker/unaligned_load}
    \caption{Img00}
    \label{fig:unlo}
\end{figure}
\FloatBarrier

\begin{lstlisting}[label=loop, caption=Unaligned accesses]

float g = static_cast< float >(inputImage[(width * height) + (y * width) + x]);
float b = static_cast< float >(inputImage[(2 * width * height) + (y * width) + x]);

\end{lstlisting}
\FloatBarrier

 






On image number 5 => no coalesced access for the 2 instructions.. because image pixels non a multiple of 128, so non aligned (threadID+pizelsOfImage) but not effect the computation, compiling with -Xptxas -dlcm=cg worst results, Changing the configuration (cudaDeviceSetCacheConfig(cudaFuncCachePreferL1);)
2) 





\begin{figure}[ht]
    \centering
    \includegraphics[width=0.7\linewidth]{profiling/darker/darker_utilization_00}
    \caption{Img05}
    \label{fig:histo}
\end{figure}
\FloatBarrier

\begin{figure}[ht]
    \centering
    \includegraphics[width=0.7\linewidth]{profiling/darker/darker_varying}
    \caption{Img05}
    \label{fig:histo}
\end{figure}
\FloatBarrier



\begin{figure}[ht]
    \centering
    \includegraphics[width=0.7\linewidth]{profiling/darker/darker_varying}
    \caption{Img05}
    \label{fig:histo}
\end{figure}
\FloatBarrier

\printbibliography 

\end{document}

			 	 